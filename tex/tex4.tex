\documentclass[dvipdfmx]{jsarticle}

\include{begin}

\section{Telnetで POSTリクエスト}

\subsection{サーバー側で準備をする}

今度は POST送信をやってみる。


まず、サーバー側に以下のようなHTMLを用意する。

\begin{lstlisting}[caption=post.html]
<!doctype html>
<html lang="ja">
  <head>
    <meta charset="utf-8"/>
    <title>Post</title>
  </head>
  <body>
    <h1>Post</h1>
    <form action="receive.php" method="post">
      <label for="name">お名前</label>
      <input type="name" name="name" id="name" />
      <input type="submit" value="送信"/>
    </form>
  </body>
</html>
\end{lstlisting}

クライアント(ブラウザ)から、以下のように GETリクエストが送られると......、

\begin{tcolorbox}
 GET /post.html HTTP/1.1 \\
 Host: localhost:8888 \\
 (空行)
\end{tcolorbox}

以下のようにレスポンスが返ってくる。

\vspace{3mm}
\includegraphics[width=11cm]{../img/32-response.png}
\vspace{3mm}

\newsavebox{\htmltext}
\setbox\htmltext=\vbox{\hsize 13cm
\begin{verbatim}
 HTTP/1.1 200 OK
 Host: localhost:8888
 Date: Sat, 21 Aug 2021 22:15:09 GMT
 Connection: close
 Content-Type: text/html; charset=UTF-8
 Content-Length: 358

 <!doctype html>
 <html lang="ja">
   <head>
     <meta charset="utf-8">
     <title>Post</title>
   </head>
   <body>
     <h1>Post</h1>
     <form action="receive.php" method="post">
       <label for="name">お名前</label>
       <input type="text" name="name" id="name" >
       <input type="submit" value="送信">
     </form>
   </body>
 </html>
\end{verbatim}
}

\fbox{\usebox{\htmltext}}

ブラウザの画面には以下のようにレンダリングされる。

\vspace{3mm}
\begin{tabular}{c|} \hline
\includegraphics[width=8cm]{../img/31-post.png} \\ \hline
\end{tabular}
\vspace{3mm}


ここでユーザーは名前を入力し、送信ボタンを押して、Webサーバーに
入力した情報を送信する。

このとき、post.html には、以下のように送り先が指定されている。

\vspace{3mm}
\begin{tcolorbox}
$<$form action=''receive.php'' method=''post''$>$
\end{tcolorbox}
\vspace{3mm}

これは、送り先が ``receive.php'' で、送信方法が ``post''送信であるという指定である。

まず、Webサーバーに、クライアント(ブラウザ)から送信された情報を受け取るための
ファイル ''receive.php'' を作成する。``test''フォルダに以下の内容で作成する。

\begin{lstlisting}[caption=receive.php]
<?php
 $name = $_POST['name'];
?>
 <!doctype html>
 <html lang="ja">
   <head>
     <meta charset="utf-8"/>
     <title>receive</title>
   </head>
   <body>
     <h1>receive</h1>
     <p>name="<?php echo $name; ?>"</p>
   </body>
 </html>
\end{lstlisting}

これは、クライアントから送られてきた ``name'' の内容を \$name という変数にセットし、
それをブラウザで表示するというものである。

実際に試してみる。
``test''フォルダでコマンドプロンプトを起動し、\fbox{ $>$ php -S localhost:8888 }
として、Webサーバーを起動する。

ブラウザで \fbox{ http://localhost:8888/post.html } にアクセスして、名前を入力して
送信する。

\vspace{3mm}
\includegraphics[width=8cm]{../img/33-post-2.png}
\vspace{3mm}

名前を入力して送信すると\dots \dots

\vspace{3mm}
\includegraphics[width=8cm]{../img/34-receive.png}
\vspace{3mm}

このようにブラウザの画面で表示される。

これを TeraTerm を使って Telnet画面でやってみる。
そのことで、ブラウザとWebサーバーでは実際にどういうやりとりが行われているかを確認する。


\subsection{TelnetでのPOST送信のしかた}

``test'' フォルダで Webサーバーが動作している状態で、TeraTerm を起動し、POST送信を
おこなうが、今度は送信文字列が長いので、TeraPad などのエディタで、送信文字列を
あらかじめ作成して、それをコピーして TeraTerm の画面に貼り付けることにする。

\begin{lstlisting}[caption=post-request.txt]
 POST /receive.php HTTP/1.1
 Host: localhost:8888
 Content-Length: 11

 name=Sasuke
\end{lstlisting}

\begin{enumerate}
 \item 今回は POST送信なので ``POST'' と指定する。\\
       また、送り先として ``receive.php'' を指定する。
 \item これは GET の場合と同じである。
 \item ここでは、ボディ部の長さを伝えている。``name=Sasuke'' で 11バイトである。
 \item 空行がヘッダ部とボディ部の区切りである。
 \item ``name'' という変数(コントロール名)に ``Sasuke'' という値(文字列)をセットしている。
\end{enumerate}

5行目の ``name=Sasuke'' は、post.html の以下の部分の記述に相当する。

\begin{tcolorbox}
 $<$input type="text" name=\underline{"name"} id="name" $>$
\end{tcolorbox}

つまり、ユーザーが入力してくれた 値(文字列) に \underline{``name''} という名前をつけて
Webサーバーに送るという意味である。

ここでは、``name'' という文字列がたくさん出てきてややこしいが、たとえば、以下のように
なっていると\dots \dots

\begin{tcolorbox}
 $<$input type="text" name=\underline{"\textsf{namae}"} id="name" $>$
\end{tcolorbox}

post-request.txt の 5行目は、以下のようになる。

\begin{tcolorbox}
 \textsf{namae}=Sasuke
\end{tcolorbox}

要するに、Webサーバーから送られてきた post.html を画面に表示し、ユーザーが
名前を入力し、送信ボタンが押されたとき、

\newsavebox{\posttext}
\setbox\posttext=\vbox{\hsize 13cm
\begin{verbatim}
 POST /receive.php HTTP/1.1
 Host: localhost:8888
 Content-Length: 11

 name=Sasuke
\end{verbatim}
}
\fbox{\usebox{\posttext}}

このようなコマンド文字列が Webサーバーに送られているのである。

そして、POST送信の特徴は、送信されるデータが ボディ部 に埋め込まれるということである。
ということは、SSL通信では暗号化されるということになる。(ヘッダ部は暗号化されない)

今回は送るデータは一つだけだが、POST送信では多くのデータを送ることができる。

\vspace{5mm}
POST送信以外に、GET送信でもデータを送ることができる。以下のようなやりかたで送信する。

\begin{tcolorbox}
 GET /receive.php\textsf{?name=Sasuke} HTTP/1.1
\end{tcolorbox}

今回は GET送信でのデータ送信は詳しくは触れない。


\subsection{TeraTermで POST送信をやってみる}

それでは、TeraTermで POST送信をやってみる。
今回は、実際のやりとりを再現するために、``ローカルエコー'' を ``OFF'' にしてみる。

今までは、TeraTerm の画面では、我々の入力した文字列と、サーバーが送り返してきた文字列を
両方とも表示していた。
しかし実際は、我々が入力した文字列はサーバーに送られてしまい、我々は見ることはできない
はずである。
それでは仕事がやりにくいので、telnetプログラムには、ユーザーが入力した文字列をサーバーに
送ると同時に、ユーザーにも表示してくれる機能がついている。
それが ``ローカルエコー'' で、それを ``ON'' にすることで実現できる。

今までは TeraTerm の画面に直接文字列を入力していたので、''ローカルエコー'' が ''ON'' の
ほうがやりやすかった。
今回は、エディタで文字列を準備しておいて、TeraTermの画面には貼り付けるだけなので、
``ローカルエコー'' を ``OFF'' にしてもできる。

``ローカルエコー'' を ``OFF'' にするには、2つの方法がある。
\begin{enumerate}
 \item 設定ファイル(TERATERM.INI) を書き変える。
 \item 一時的に変更する。
\end{enumerate}

今回は、``一時的に変更する'' のほうでやってみる。

手順をまとめると以下になる。

\vspace{3mm}

\newsavebox{\itemtext}
\setbox\itemtext=\vbox{\hsize 15cm
\begin{enumerate}
 \item TeraTermを起動する。
 \item ``設定'' --- ``端末'' を選択して ``端末の設定''画面で、``ローカルエコー''のチェックをはずす。
 \item ``post-request.txt'' の内容をコピーする。
 \item ``編集'' --- ``貼り付け'' を選択。
 \item 貼り付ける文字列の確認画面が開くので、それをOKする。
 \item 黒い画面になるので、``$<$Enter$>$''キーを押す。
 \item サーバーから文字列が送られてくる。
\end{enumerate} 
}

\fbox{\usebox{\itemtext}}

以下、ひとつひとつの手順を確認する。


\newpage
\textsf{1.} TeraTermを起動する。

\vspace{3mm}
\includegraphics[width=10cm]{../img/41-post-connect.png}
\vspace{3mm}

\textsf{2.} ``設定'' --- ``端末'' を選択して ``端末の設定''画面で、``ローカルエコー''のチェックをはずす。

\vspace{3mm}
\includegraphics[width=10cm]{../img/42-localecho-off.png}
\vspace{3mm}

\textsf{3.} ``post-request.txt'' の内容をコピーする。

\vspace{3mm}
\includegraphics[width=10cm]{../img/43-post-copy.png}
\vspace{3mm}

余分な''空行''などコピーしないように ``Sasuke'' までをコピーする。

\newpage
\textsf{4.} TeraTermの黒い画面で ``編集'' --- ``貼り付け''

\vspace{3mm}
\includegraphics[width=10cm]{../img/44-edit-paste.png}
\vspace{3mm}

\textsf{5.} 貼り付ける文字列の確認画面が開くので、それをOKする。

\vspace{3mm}
\includegraphics[width=10cm]{../img/45-paste-confirm.png}
\vspace{3mm}

\newpage
\textsf{6.} 黒い画面になるので、``$<$Enter$>$''キーを押す。

こちらから貼り付けた文字列はサーバーへ送られたので、画面には出ていない。
これが ``ローカルエコー・オフ'' ということである。

\vspace{3mm}
\includegraphics[width=10cm]{../img/46-black.png}
\vspace{3mm}

\textsf{7.} サーバーから文字列が送られてくる。

\vspace{3mm}
\includegraphics[width=10cm]{../img/47-post-result.png}
\vspace{3mm}

送られてきた文字列は、以下である。

\newsavebox{\htmlbox}
\setbox\htmlbox=\vbox{\hsize 13cm
\begin{verbatim}
HTTP/1.1 200 OK
Host: localhost:8888
Date: Sun, 22 Aug 2021 03:15:13 GMT
Connection: close
X-Powered-By: PHP/8.0.9
Content-type: text/html; charset=UTF-8

<!doctype html>
<html lang="ja">
  <head>
    <meta charset="utf-8"/>
    <title>receive</title>
  </head>
  <body>
    <h1>receive</h1>
    <p>name="Sasuke"</p>
  </body>
</html>
\end{verbatim}
}
\fbox{\usebox{\htmlbox}}

ヘッダ部は GET送信のときとそんなに変りはない。
5行目で PHP で処理したことが明記されている。

ボディ部のほうは、``receive.php''の内容であるが、\verb!<?php ... ?>! で記述された
部分はない。

特に、\verb!<?php echo $name; ?>! で記述された部分は 文字列''Sasuke''
に置き変わっている。









\include{end}


%% 修正時刻: Sun Aug 22 13:52:28 2021
