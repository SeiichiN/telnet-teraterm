\documentclass[dvipdfmx]{jsarticle}

\include{begin}

\section{Telnetで GETリクエスト}

\subsection{ブラウザは何をやっているのか?}

現在、\textsf{test} フォルダを公開フォルダとして Webサーバーが動作している。

ブラウザから \textsf{http://localhost:8888} とすると、このフォルダにある
\textsf{index.html} を表示させることができる。

このとき、ブラウザはWebサーバーにどういう働きかけをしているのか?

また、Webサーバーはどういう返答をしているのか?

ブラウザのやっていることを \textsf{Telnet} でやってみることができる。

\subsection{GETリクエスト}

ブラウザから、あるサイトにアクセスして、そのページを表示させたいとする。

これをあるがままに言うと、ブラウザがあるサイトにアクセスして、

\framebox[10cm][c]{このページのHTMLをダウンロードさせてくれ}

という要求をおこなっていることになる。

これを \textgt{GETリクエスト} という。

\textgt{GETリクエスト} のやり方は決っていて、以下のようなコマンドを送る。

\begin{tcolorbox}
 \verb!GET / HTTP/1.1! \\
 \verb!Host: localhost:8888! \\
 (空行)
\end{tcolorbox}

1行目は、GETリクエストであることを示す。 
``/'' は、パスをあらわす。''/index.html'' を指定したことと同じ。 
''HTTP.1.1'' は、HTTPの1.1バージョンで通信するという意味。

2行目は、相手ホスト名を指定している。
80番ポート(普通はこのポート) なら省略できるが、我々はさきほど、このWebサーバーを
8888番ポートで起動したから、このポートを指定している。

そして、3行目は、\textgt{空行} を送る。これは絶対必要。

最後に $<$Enter$>$ を送る。

この空行までを\textgt{ヘッダ部}という。これが GETリクエストの決まりである。

これを TeraTerm でやってみる。

\subsection{TeraTermでGetリクエスト}

TeraTerm を起動する。

``ホスト'' に \textsf{localhost} と指定する。

``サービス'' は \textsf{Telnet} にチェックを入れる。

''TCPポート'' は \textsf{8888} と指定する。

ほかはそのままで、``OK''とする。

\vspace{3mm}
\includegraphics{../img/25-get-01.png}
\vspace{3mm}

黒い画面があらわれるので、以下の内容を黒い画面に入力する。

\begin{tcolorbox}
 \verb!GET / HTTP/1.1! \\
 \verb!Host: localhost:8888! \\
 (空行)
\end{tcolorbox}

大文字、小文字に気をつけて入力する。

\vspace{3mm}
\includegraphics[width=14cm]{../img/26-get-02.png}
\vspace{3mm}

そして $<$Enter$>$ キーを押すと、以下のように出力される。

\vspace{3mm}
\includegraphics[width=14cm]{../img/27-get-03.png}
\vspace{3mm}

空行の下の ``HTTP...''以下は Webサーバーからの応答(レスポンス)である。

\begin{itemize}
 \item \textsf{HTTP/1.1} --- HTTPプロトコルのバージョン1.1 で応答するということ。
 \item \textsf{200 OK} ---  正常な処理だということ。
 \item \textsf{Host: localhost:8888} --- サーバー側のホスト名とポート番号。
 \item 日付 -- ''\textsf{GMT}'' となっているのは、グリニッジ標準時のこと。日本よりも9時間遅い。
 \item \textsf{Connection: close} --- この応答で接続が閉じられたことを表す。
 \item \textsf{Content-Type: text/html; charset=UTF-8} --- 空行の後に送る文字列(ボディ部)は
 HTMLで、文字コードは UTF-8 であることを (ブラウザに) 伝えている。
 \item \textsf{Content-Length: 210} --- 空行の後に送る文字列は 210文字であることを示している。\\
※ 210文字というのは、本当は 159文字で、この画像では消しているが、$<$/html$>$ の後に注釈の
文字列が続いていたのである。
\end{itemize}

ここまでが ''\textgt{ヘッダ部}'' で、''\textsf{空行}'' に区切られて、以下 ''\textsf{ボディ部}''が
続く。

ボディ部はブラウザが受け取って処理をする。つまり、文字や画像をブラウザの画面に表示するのである。
それを \textsf{レンダリング(描画)} という。

\subsection{さまざまな GET リクエスト}

Webサーバーに送る GET リクエストでは、``/'' のみを指定したが、それ以外にたとえば、``/menu.html''
などとファイル名(パス)を指定することが多い。

もし、存在しないファイル名を指定すると、どうなるか?

これは、存在しないファイル ''about.html'' を指定した例である。

\vspace{3mm}
\includegraphics[width=15cm]{../img/28-not-found.png}
\vspace{3mm}

サーバーからの応答の1行目に、

\begin{tcolorbox}
 HTTP/1.1 404 Not Found
\end{tcolorbox}

とある。''404エラー'' と呼ばれる応答例である。
これをブラウザで表示すると、以下のようになる。

\vspace{3mm}
\includegraphics[width=10cm]{../img/29-not-found-2.png}
\vspace{3mm}

\subsection{ブラウザのやっていること}

ブラウザは、ユーザーが URL欄にアドレスを入力したら、そのアドレスに対して GETリクエストを
送信する。

Webサーバーは、クライアントであるブラウザから URLを受け取って、そのファイルが存在するなら
``200'' という番号をヘッダ部に書き、ファイル本体をボディ部に書いて、ブラウザに送る。

もしそのファイルが存在しなかったら、``404''という番号をヘッダ部に書き、``Not Found'' と表示させるための文字列をボディ部に書いて、ブラウザに送る。

Webサーバーから送られてきた応答のうち、ヘッダ部は、Webサーバーとブラウザとの暗黙のやりとり
であり、ブラウザには直接表示されない。
確認するためには、ブラウザのデベロッパーツールの''ネットワーク''タブを見るとよい。
ボディ部はブラウザが画面にレンダリングするための情報である。
デベロッパーツールの''インスペクター''タブで確認できる。

ボディ部は HTML という約束事にしたがって書かれている。
(HTMLのなかにも $<$head$>$部と$<$body$>$部があるが、混乱しないように。)




\include{end}

%% 修正時刻: Sun Aug 22 06:47:04 2021
